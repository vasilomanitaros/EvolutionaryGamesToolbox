\chapter{Εισαγωγή}
Η παρούσα εργασία πραγματεύεται την υλοποίηση του εξελικτικού πρωταθλήματος \textit{Axelrod}, το οποίο έχει ως θεωρητική βάση το επαναλαμβανόμενο δίλημμα του φυλακισμένου (Iterated Prisoner's Dilemma - IPD). Η εργασία αυτή αποτελεί το πέμπτο παραδοτέο στο πλαίσιο του μαθήματος της Θεωρίας Παιγνίων, και ακολουθεί προηγούμενες εργασίες στις οποίες έχουν υλοποιηθεί:(α) το περιβάλλον του παιχνιδιού,(β) οι βασικές στρατηγικές που χρησιμοποιούνται στα πειράματα.
Για την καλύτερη κατανόηση, είναι σημαντικό να εξηγηθεί πρώτα τι είναι το δίλημμα του φυλακισμένου και το πρωτάθλημα Axelrod.

\subsubsection*{Το Δίλημμα του Φυλακισμένου}
Το κλασικό δίλημμα του φυλακισμένου είναι ένα υποθετικό σενάριο συνεργασίας και προδοσίας μεταξύ δύο παικτών. Κάθε παίκτης έχει δύο επιλογές: να συνεργαστεί (\textbf{C}) ή να προδώσει (\textbf{D}). Οι αποδόσεις (\textbf{payoffs}) κάθε παίκτη εξαρτώνται από τον συνδυασμό των επιλογών τους και μπορούν να περιγραφούν με τέσσερις βασικές τιμές:
\begin{itemize}
	\item \textbf{T (Temptation to Defect)} η ανταμοιβή που παίρνει κάποιος όταν προδίδει ενώ ο άλλος συνεργάζεται.
	\\
	\item \textbf{R (Reward for Mutual Cooperation)}: η ανταμοιβή που παίρνουν και οι δύο όταν συνεργάζονται.
	\\
	\item \textbf{P (Punishment for Mutual Defection)}: η τιμωρία που λαμβάνουν όταν και οι δύο προδίδουν.
	\\
	\item 
	\textbf{S (Sucker's Payoff)}: η "ζημιά" που υφίσταται κάποιος όταν συνεργάζεται ενώ ο άλλος τον προδίδει
	
\end{itemize}
Ο πίνακας αποδόσεων (\textit{payoff matrix}) έχει ως εξής:
\begin{table}[h]
	\centering
	
	\vspace*{2em}
	\begin{tabular}{cc|c|c|}
		& \multicolumn{1}{c}{} & \multicolumn{2}{c}{P2} \\
		& \multicolumn{1}{c}{} & \multicolumn{1}{c}{C (Συνεργασία)} & \multicolumn{1}{c}{D (Προδοσία)} \\\cline{3-4}
		\multirow{2}*{P1} & C & R, R & S, T \\\cline{3-4}
		& D & T, S & P, P \\\cline{3-4}
	\end{tabular}
	\caption{Payoff Matrix}
\end{table}\\
Οι τυπικές τιμές που χρησιμοποιούνται στο IPD είναι:
\(T = 5, R = 3, P = 1, S = 0\)
\\
\clearpage
Για να υπάρχει πραγματικό «δίλημμα», πρέπει να ισχύουν δύο βασικές \textbf{λογικές συνθήκες:
}
\begin{enumerate}
	\item \(\bm{S<P<R<T}\)\\
	Δηλαδή: είναι χειρότερο να σε προδώσουν ενώ συνεργάζεσαι \((S)\), από το να προδώσετε και οι δύο \((P)\), και προτιμότερο να συνεργαστείτε και οι δύο \((R)\), αλλά ο μεγαλύτερος πειρασμός είναι να προδώσεις όταν ο άλλος συνεργάζεται \((T)\).
	\\
	\item \(\bm{S + T < 2R}\) \\Αυτή η επιπλέον συνθήκη εξασφαλίζει ότι η\textbf{ συνεργασία είναι συλλογικά προτιμότερη} από την εναλλαγή μεταξύ προδοσίας και συνεργασίας. Δηλαδή, αν οι παίκτες εναλλάσσονται μεταξύ \(C \) και \(D\), οι αποδόσεις είναι χειρότερες από τη σταθερή συνεργασία.
	\\
	\\
	Το δίλημμα έγκειται στο γεγονός ότι, ενώ η \textbf{αμοιβαία συνεργασία} οδηγεί σε καλύτερη συνολική απόδοση και για τους δύο \((R)\), η \textbf{προδοσία είναι η ατομικά καλύτερη επιλογή}, ανεξάρτητα από την κίνηση του άλλου. Αυτό συμβαίνει επειδή:
	\begin{itemize}
		\item Αν ο άλλος συνεργάσει, το να προδώσεις σου δίνει \(\bm{T > R}\)
		\item Αν ο άλλος προδώσει, το να προδώσεις σου δίνει \(\bm{P > S}\)
		
		
	\end{itemize}
	Έτσι, η προδοσία είναι \textbf{το μοναδικό σημείο ισορροπίας Nash} στο παίγνιο μίας φοράς. Ωστόσο, όπως θα δούμε στη συνέχεια, όταν το παιχνίδι επαναλαμβάνεται, μπορούν να προκύψουν πολύ διαφορετικές δυναμικές.
	
\end{enumerate}
\subsection*{Επαναλαμβανόμενο Δίλημμα του Φυλακισμένου (IPD)}
Επαναλαμβανόμενο Δίλημμα του Φυλακισμένου (Iterated Prisoner's Dilemma)

Σε αντίθεση με το απλό παίγνιο που παίζεται μόνο μία φορά, \textbf{στο επαναλαμβανόμενο δίλημμα του φυλακισμένου}, οι δύο παίκτες παίζουν πολλές φορές μεταξύ τους, και μπορούν να βασίσουν την απόφασή τους για την επόμενη κίνηση στις προηγούμενες κινήσεις του αντιπάλου. Αυτό επιτρέπει την εμφάνιση πιο πολύπλοκων στρατηγικών, όπως η \textbf{Tit for Tat} (ξεκινά με συνεργασία και έπειτα αντιγράφει την προηγούμενη κίνηση του αντιπάλου), και δημιουργεί δυναμική αλληλεπίδρασης και «μνήμης».

\subsection*{Το Πρωτάθλημα Axelrod}
Το \textbf{Πρωτάθλημα Axelrod} ήταν ένα διάσημο υπολογιστικό πείραμα που διεξήγαγε ο πολιτικός επιστήμονας \textbf{Robert Axelrod} τη δεκαετία του 1980. Ο Axelrod προσκάλεσε ερευνητές να υποβάλουν στρατηγικές για το επαναλαμβανόμενο δίλημμα του φυλακισμένου. Οι στρατηγικές αυτές διαγωνίστηκαν μεταξύ τους σε ένα τουρνουά όπου η κάθε στρατηγική έπαιζε επαναλαμβανόμενα παιχνίδια με όλες τις υπόλοιπες, συγκεντρώνοντας βαθμούς με βάση τα αποτελέσματα κάθε αναμέτρησης.
Το τουρνουά του Axelrod ανέδειξε τη σημασία της συνεργασίας και της «τιμωρίας με μέτρο» ως αποτελεσματική στρατηγική. Χαρακτηριστικά, η απλή στρατηγική \textbf{Tit for Tat} αναδείχθηκε ως μία από τις πιο επιτυχημένες, αναδεικνύοντας τη δύναμη της αμοιβαιότητας στη σταθερή συνεργασία.
\clearpage
\subsection*{Σκοπός και Δομή της Παρούσας Εργασίας
}Η εργασία επικεντρώνεται στην \textbf{εξελικτική εκδοχή} του τουρνουά, όπου οι στρατηγικές «επιβιώνουν» ή «αντικαθίστανται» ανάλογα με την απόδοσή τους. Αναλυτικότερα, η εργασία χωρίζεται σε δύο βασικές προσεγγίσεις
\begin{enumerate}
	\item \textbf{Imitation Dynamics}\\ Σε αυτό το μοντέλο, σε κάθε γενιά ένας αριθμός παικτών αντικαθιστά τη στρατηγική του με εκείνη κάποιου άλλου που έχει καλύτερη απόδοση (Score). Ο πληθυσμός εξελίσσεται μιμούμενος τις καλύτερες στρατηγικές.
	\item 	\textbf{Fitness Dynamics}: Σε αυτή την προσέγγιση, η πιθανότητα να επιβιώσει ή να αναπαραχθεί μια στρατηγική εξαρτάται από το μέσο Score που συγκεντρώνει. Πρόκειται για πιο αναλυτική, συνεχής προσέγγιση εμπνευσμένη από τη βιολογική εξέλιξη.
	
\end{enumerate}
Η βασική διαφορά μεταξύ των δύο μοντέλων έγκειται στον τρόπο που οι στρατηγικές διαδίδονται στον πληθυσμό:
\begin{itemize}
	\item Στο \textbf{Fitness Dynamics}, η αλλαγή εξαρτάται από το \textbf{συνολικό αποτέλεσμα} κάθε στρατηγικής σε κάθε γενιά.
	\item Στο \textbf{Imitation Dynamics}, επικεντρωνόμαστε στους \textbf{καλύτερους παίκτες}, και οι αλλαγές είναι διακριτές και συμβαίνουν σε \textbf{συγκεκριμένο αριθμό} παικτών ανά γενιά.
	
\end{itemize}Στη θεωρητική ανάλυση του Fitness Dynamics, επιβεβαιώνονται τα αποτελέσματα των \cite{paper} Mathieu et al., ενώ στο Imitation Dynamics γίνεται υλοποίηση της \textbf{αλυσίδας Markov} για όλες τις πιθανές κατανομές πληθυσμού, ανάλογα με το μέγεθος πληθυσμού και τη δυναμική μετάβασης.
