\chapter{Discussion}
Μέσα από την θεωρητική ανάλυση του Fitness Dynamics παρατηρούμε την ακριβή επαλήθευση των αποτελεσμάτων του paper καθώς και την επαλήθευσή τους σε πραγματικό χρόνο από την προσομοίωση. 
\\

Σε κάθε περίπτωση οι αρχικές συνθήκες παίζουν πολύ σημαντικό ρόλο και δόθηκε αρκετός χρόνος στην επίτευξη των ακριβών αποτελεσμάτων. Στη συνέχεια από το Imitation Dynamics παρατηρούμε την σύγκλιση των πληθυσμών προς σταθερές καταστάσεις μέσα από τις αλυσίδες Markov και πώς μερικές φορές από μια αρχική κατάσταση είμαστε βέβαιοι ότι στο τέλος θα οδηγηθούμε σε μια συγκεκριμένη τελική ενώ άλλες όχι. Τα βήματα προς την τελική κατάσταση από το Simulation των Imitation Dynamics εξαρτώνται και αυτά από διάφορους μηχανισμούς όπως την επιλογή βέλτιστης στρατηγικής και τον μηχανισμό επιλογής των παικτών προς αλλαγή. Σε αυτό το σημείο και καθώς δόθηκε αρκετή ελευθερία στην μοντελοποίηση του συστήματος τα αποτελέσματα μας ενδέχεται να διαφέρουν σε σχέση με τις υπόλοιπες ομάδες, ενώ καθώς ο τρόπος επιλογής παικτών προς αλλαγή περιλαμβάνει στοχαστικότητα τα figures που συμπεριλάβαμε ενδέχεται να μη παραχθούν ολόιδια με τα scripts ασχέτως αν ο μηχανισμός παραμένει κοινός.\\ 
Τέλος, η ανάλυση αυτή μπορεί να δώσει χρήσιμα συμπεράσματα τόσο για τα δυναμικά φαινόμενα στα εξελικτικά πρωταθλήματα ενώ σημαντικό ερευνητικό ενδιαφέρον θα παρουσίαζε η εφαρμογή της παραπάνω ανάλυσης και στα άλλα 2 διαφορετικά κοινωνικά διλλήματα τα οποία είναι οι Ιέρακες και Περιστερές και το Κυνήγι του Ελαφιού με αντίστοιχες προσαρμογές.